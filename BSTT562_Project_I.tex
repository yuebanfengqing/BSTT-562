\documentclass[12pt,]{article}
\usepackage{lmodern}
\usepackage{amssymb,amsmath}
\usepackage{ifxetex,ifluatex}
\usepackage{fixltx2e} % provides \textsubscript
\ifnum 0\ifxetex 1\fi\ifluatex 1\fi=0 % if pdftex
  \usepackage[T1]{fontenc}
  \usepackage[utf8]{inputenc}
\else % if luatex or xelatex
  \ifxetex
    \usepackage{mathspec}
  \else
    \usepackage{fontspec}
  \fi
  \defaultfontfeatures{Ligatures=TeX,Scale=MatchLowercase}
\fi
% use upquote if available, for straight quotes in verbatim environments
\IfFileExists{upquote.sty}{\usepackage{upquote}}{}
% use microtype if available
\IfFileExists{microtype.sty}{%
\usepackage{microtype}
\UseMicrotypeSet[protrusion]{basicmath} % disable protrusion for tt fonts
}{}
\usepackage[margin = 1in]{geometry}
\usepackage{hyperref}
\hypersetup{unicode=true,
            pdftitle={BSTT562 Project I},
            pdfauthor={Ruizhe Chen, Hesen Li},
            pdfborder={0 0 0},
            breaklinks=true}
\urlstyle{same}  % don't use monospace font for urls
\usepackage{natbib}
\bibliographystyle{plainnat}
\usepackage{longtable,booktabs}
\usepackage{graphicx,grffile}
\makeatletter
\def\maxwidth{\ifdim\Gin@nat@width>\linewidth\linewidth\else\Gin@nat@width\fi}
\def\maxheight{\ifdim\Gin@nat@height>\textheight\textheight\else\Gin@nat@height\fi}
\makeatother
% Scale images if necessary, so that they will not overflow the page
% margins by default, and it is still possible to overwrite the defaults
% using explicit options in \includegraphics[width, height, ...]{}
\setkeys{Gin}{width=\maxwidth,height=\maxheight,keepaspectratio}
\IfFileExists{parskip.sty}{%
\usepackage{parskip}
}{% else
\setlength{\parindent}{0pt}
\setlength{\parskip}{6pt plus 2pt minus 1pt}
}
\setlength{\emergencystretch}{3em}  % prevent overfull lines
\providecommand{\tightlist}{%
  \setlength{\itemsep}{0pt}\setlength{\parskip}{0pt}}
\setcounter{secnumdepth}{5}
% Redefines (sub)paragraphs to behave more like sections
\ifx\paragraph\undefined\else
\let\oldparagraph\paragraph
\renewcommand{\paragraph}[1]{\oldparagraph{#1}\mbox{}}
\fi
\ifx\subparagraph\undefined\else
\let\oldsubparagraph\subparagraph
\renewcommand{\subparagraph}[1]{\oldsubparagraph{#1}\mbox{}}
\fi

%%% Use protect on footnotes to avoid problems with footnotes in titles
\let\rmarkdownfootnote\footnote%
\def\footnote{\protect\rmarkdownfootnote}

%%% Change title format to be more compact
\usepackage{titling}

% Create subtitle command for use in maketitle
\newcommand{\subtitle}[1]{
  \posttitle{
    \begin{center}\large#1\end{center}
    }
}

\setlength{\droptitle}{-2em}

  \title{BSTT562 Project I}
    \pretitle{\vspace{\droptitle}\centering\huge}
  \posttitle{\par}
    \author{Ruizhe Chen, Hesen Li}
    \preauthor{\centering\large\emph}
  \postauthor{\par}
      \predate{\centering\large\emph}
  \postdate{\par}
    \date{November 29, 2018}


\begin{document}
\maketitle

\section{Abstract}\label{abstract}

In this project, we want to reproduce the results presented in Table 3
Interlaboratory Data for Cadmium from \citet{Bhaumik2005}. \#
Introduction

A key characteristics of the data is its heteroscedasticity.

\section{Data}\label{data}

This data is the experimental data fro cadmium from an interlaboratory
study conducted by Ford Motor Company (J. Phillips, personal
communication). These data were generated as part of a blind
interlaboratory study of laboratories that hold Michigan State Drinking
Water Certifications for the parameters tested. Samples were prepared by
an independent source, randomized, and submitted on a weekly basis over
a 5-week period. Cadmium was analyzed by inductively coupled plasma
atomic emissions spectroscopy using EPA method 200.7. The data set used
for this example comprised five replicates at each of three
concentrations (0, 20 and 100 \(\mu g/L\)) in each of the \(q = 5\)
laboratories. Using the first replicate from the first three
laboratories as the new measurement (i.e., \(q' = 3\)), we would like to
reproduce the results of point estimates, variances, confidence
intervals and simulated confidence levels for the true concentrations in
each of the three cases. The data is displayed as follows:

\begin{longtable}[]{@{}rrrrr@{}}
\caption{Interlaboratory Data for Cadmium (ug / L)}\tabularnewline
\toprule
Lab & Replication & 0 & 20 & 100\tabularnewline
\midrule
\endfirsthead
\toprule
Lab & Replication & 0 & 20 & 100\tabularnewline
\midrule
\endhead
1 & 1 & -3.000 & 10.000 & 92.000\tabularnewline
1 & 2 & 4.000 & 20.000 & 100.000\tabularnewline
1 & 3 & -4.000 & 17.200 & 97.800\tabularnewline
1 & 4 & 3.000 & 24.000 & 100.000\tabularnewline
1 & 5 & 3.100 & 19.100 & 109.000\tabularnewline
2 & 1 & -0.060 & 17.815 & 90.455\tabularnewline
2 & 2 & 0.010 & 17.305 & 87.610\tabularnewline
2 & 3 & 0.115 & 16.570 & 85.550\tabularnewline
2 & 4 & -0.055 & 17.360 & 89.925\tabularnewline
2 & 5 & 0.340 & 18.120 & 90.070\tabularnewline
3 & 1 & -7.400 & 27.100 & 107.400\tabularnewline
3 & 2 & -2.100 & 19.400 & 108.100\tabularnewline
3 & 3 & -11.400 & 9.000 & 83.800\tabularnewline
3 & 4 & -11.100 & 10.500 & 81.900\tabularnewline
3 & 5 & -1.400 & 19.300 & 94.200\tabularnewline
4 & 1 & 1.000 & 21.000 & 96.000\tabularnewline
4 & 2 & -2.126 & 16.049 & 90.650\tabularnewline
4 & 3 & 0.523 & 16.082 & 89.388\tabularnewline
4 & 4 & -2.000 & 17.000 & 91.000\tabularnewline
4 & 5 & -0.551 & 15.489 & 85.867\tabularnewline
5 & 1 & 0.000 & 18.000 & 91.000\tabularnewline
5 & 2 & 0.000 & 19.000 & 101.000\tabularnewline
5 & 3 & 0.000 & 19.000 & 102.000\tabularnewline
5 & 4 & -1.000 & 18.700 & 92.700\tabularnewline
5 & 5 & 0.038 & 19.790 & 99.884\tabularnewline
\bottomrule
\end{longtable}

\section{Method}\label{method}

To measure the true concentration of an analyte, \(x\), the traditional
way is to propose a simple linear calibration model:

\begin{equation}
y = \alpha + \beta x + e
\end{equation}

with normal assumption on errors. But as the data is heterogeneous, this
models fails to explain the increasing measurement variation with
increasing analyte concentration commonly observed in analytic data. To
overcome this drawback of this simple model, we propose a log-normal
model:

\begin{equation}
y = x e^\eta
\end{equation}

\section{Analysis}\label{analysis}

We choose the Method of Moments to estimate the model parameters. The
Method of Moments is straightforward and easy to implement when
observations with lower concentrations are available, which happens to
be the case with our Interlaboratory Data for Cadmium.

Why do we choose the Method of Moments: A good property of MOM: the
estimates obtained by MOM are asymptotically efficient.

Data Separation Technique

\subsection{The model}\label{the-model}

\begin{equation}
y_{ijk} = \alpha_i + \beta_j x_j e^{\eta_{jik}} + e_{jik}
\end{equation}

Note that for this particular model, when \(x\) is 0 or near 0, the
model is reduced to:

\begin{align}
y_{ijk} &= \alpha_i + \beta_j \times 0 \times e^{\eta_{jik}} + e_{jik} \\
y_{ijk} &= \alpha_i + e_{jik} 
\end{align}

\subsection{The Estimation Procedures}\label{the-estimation-procedures}

\subsubsection{(Partial) Estimation Using Low-Concentration
Observations}\label{partial-estimation-using-low-concentration-observations}

We estimate the variance \(\sigma_e^2\) of the additive errors
\(e_{ijk}\)'s (of the \(k\)th measurement at the \(j\)th concentration
level in the \(i\) th laboratory,
\(i=1,2,...,q, j = 1,2,...,r, k=1,2,...,N_{ij}\)) that are present
primarily at low-level concentrations and the calibration parameter
\(\alpha_i\) for the \(i\)th laboratory using low-concentration
observations.

Assuming that observations corresponding to zero or near-zero are
available. Let \(y_{i0k}\) be the \(k\)th measured concentration
corresponding to the true low-level concentration from laboratory \(i\)
and \(n_{i0}\) is the number of samples with true low-level
concentrations submitted to laboratory \(i\):

\begin{enumerate}
\def\labelenumi{(\arabic{enumi})}
\tightlist
\item
  Estimate \(\sigma_e^2\) by the variance of the observations with zero
  or near-zero concentrations (\(y_{i0k}\)) by
\end{enumerate}

\begin{equation}
\hat{\sigma}_e^2 = \frac{1}{q} \displaystyle \sum_{i = 1}^{q} (\frac{\sum_{k=1}^{n_{i0}}(y_{i0k}-\bar{y}_{i0})^2}{n_{i0}-1})
\end{equation}

\begin{enumerate}
\def\labelenumi{(\arabic{enumi})}
\setcounter{enumi}{1}
\tightlist
\item
  Estimate \(\alpha_i\) from the observations with zero or near-zero
  concentrations (\(y_{i0k}\)) by
\end{enumerate}

\begin{equation}
\hat{\alpha_i} = \frac{\sum_{k=1}^{n_{i0}}y_{i0k}}{n_{i0}}
\end{equation}

\subsubsection{(Partial) Estimation Using Higher-Concentration
Observations}\label{partial-estimation-using-higher-concentration-observations}

Let \$z\_\{ijk\} = \frac{(y_{ijk}-\alpha_i)}{x_j}

\section{Table/Discussion}\label{tablediscussion}

\renewcommand\refname{References}
\bibliography{Reference.bib}


\end{document}
